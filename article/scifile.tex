

\documentclass[12pt]{article}
\usepackage{scicite}
\usepackage{times}
\usepackage{amsmath,amsfonts,amssymb,amsthm,epsfig,epstopdf,titling,url,array}


\theoremstyle{plain}
\newtheorem{thm}{Theorem}[section]
\newtheorem{lem}[thm]{Lemma}
\newtheorem{prop}[thm]{Proposition}
\newtheorem*{cor}{Corollary}

\theoremstyle{definition}
\newtheorem{defn}{Definition}[section]
\newtheorem{conj}{Conjecture}[section]
\newtheorem{exmp}{Example}[section]

\theoremstyle{remark}
\newtheorem*{rem}{Remark}
\newtheorem*{note}{Note}






\topmargin 0.0cm
\oddsidemargin 0.2cm
\textwidth 16cm 
\textheight 21cm
\footskip 1.0cm

\newenvironment{sciabstract}{%
\begin{quote} \bf}
{\end{quote}}
\renewcommand\refname{References and Notes}

\newcounter{lastnote}
\newenvironment{scilastnote}{%
\setcounter{lastnote}{\value{enumiv}}%
\addtocounter{lastnote}{+1}%
\begin{list}%
{\arabic{lastnote}.}
{\setlength{\leftmargin}{.22in}}
{\setlength{\labelsep}{.5em}}}
{\end{list}}


% Include your paper's title here

\title{Wikipedia Recommender System with serendipity} 


\author
    {
      Barszezak Yoann,Bricout Raphael, David Nicolas, Dupre Remi,\\
      Fouche Aziz, Gourdel Garance, Kaddar Younesse, Mallem Maher,\\
      \\
      \normalsize{Department of Computer science, ENS Paris-Saclay}\\
    }

 

    \date{November 2017}



%%%%%%%%%%%%%%%%% END OF PREAMBLE %%%%%%%%%%%%%%%%



\begin{document} 

% Double-space the manuscript.

\baselineskip24pt

% Make the title.

\maketitle 



% Place your abstract within the special {sciabstract} environment.

\begin{sciabstract}
  TO DO : Abstract
\end{sciabstract}





\section{Introduction}


\section{Problem Overview}




\section{Retrivial of Candidate articles}

\subsection{Neighbourhood}
\subsubsection{Naive distance}

\begin{defn}
  Let $A_1$ and $A_2$ be two articles. We call $C_1$ (resp. $C_2$) the set of categories linked to $A_1$ (resp. $A_2$).
  We define the naive distance between $A_1$ and $A_2$ the following quantity:\\
  \begin{equation*}
    d(A_1,A_2) = \frac{min(Card(C_1),Card(C_2))}{Card(C_1 \cap C_2)}
  \end{equation*}
\end{defn}


\subsubsection{Ontology basis}

\subsection{serendipity}



\section{Candidate Ranking}

\subsection{possible input in neural network}
\subsubsection{Category vector}
\subsubsection{Word2Vect and Wikidata}

\subsection{Wide and deep Neural Network}
\subsubsection{Wide : memorization/focus}
\subsubsection{Deep : generalization/serendipity}

\section{System architecture}


\section{Exprimentation}


\section{State of the Art}



\bibliography{scibib}

\bibliographystyle{Science}

\end{document}




















